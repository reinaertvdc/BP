\chapter{Dutch Summary} \label{chap:summary}

Ubiquitous computing en augmented reality zijn twee snel opkomende technologieën. De dalende prijzen van smartphones en andere krachtige hardware creëeren een breed platform dat door software als digitale assistenten en Pokémon Go gebruikt wordt om het grote publiek op een betaalbare manier tot deze concepten te introduceren. Apparaten zoals de Microsoft Hololens, die toelaten om op korte tijd prototypes te creëeren voor beide technologieën, zijn nu beschikbaar voor ontwikkelaars.

Iedere opkomende technologie brengt zijn eigen nieuwe uitdagingen met zich mee, en een belangrijke uitdaging binnen ubiquitous computing betreft het feit dat acties in de echte wereld typisch onomkeerbaar zijn, in tegenstelling tot wat voor traditionele computers doogaans het geval is. Daarom wordt feedforward, oftewel het duidelijk maken aan de gebruiker wat het resultaat van een actie zal zijn voor hij deze uitvoert, niet langer alleen waardevol maar ook werkelijk onmisbaar. Wij stellen dat AR een uitstekend middel kan zijn om deze feedforward te voorzien.

Hoewel we veel verschillende toepassingen voor feedforward door middel van AR overwogen hebben binnen deze thesis, hebben we besloten om ons te focusen op het overbrengen aan de gebruiker welke lichten aan zullen gaan wanneer hij een bepaalde lichtschakelaar indrukt. Dit probleem is herkenbaar en eenvoudig te begrijpen, wat experimentatie vergemakkelijkt, en biedt gelijktijdig ook voldoende complexiteit.

Na de initiële literatuurstudie startten we ons onderzoek met een exploratie van de design space, gebaseerd op een online enquête met 34 respondenten. In deze enquête evalueerden we zeven verschillende AR visualisaties die helpen bij het linken van lichten en schakelaars. We concludeerden dat plattegronden en kleuren de meest veelbelovende benaderingen zijn. Hoewel de twee richtingaanwijzers, zijnde lijnen en bogen, niet bijzonder goed scoorden, weten we dit aan de statische en low-fidelity hoedanigheid van de enquête, en vermoedden we dat ze veel beter zouden presteren in een meer high-fidelity experiment.

Onze laatste stap was om feedforward via plattegronden, kleuren en richtingaanwijzers te implementeren op een Microsoft HoloLens, en om onze implementatie te evalueren in een informele user study met zeven participanten. De resultaten van deze studie versterkten onze eerdere vermoedens dat plattegronden en kleuren uitstekende benaderingen zijn om het linken van lichten en schakelaars te ondersteunen, maar de richtingaanwijzers scoorden gemengde resultaten, waarbij een deel van de negativiteit geweten werd aan onze specifieke implementatie en een deel aan de voorziene scenarios. 

Mogelijk toekomstig werk omvat het ontwikkelen van een verbeterd prototype en het uitvoeren van een formele high-fidelity user test, idealiter met meer deelnemers en in gevarieerdere scenarios. We vermoeden dat zulk een experiment alsnog zou aantonen dat richtingaanwijzers effectief zijn. Verder hebben we onze experimentatie gelimiteerd tot standaard lichten en lichtschakelaars, en dus blijft onderzoek naar de toepassingen van onze visualisaties met complexere apparaten, zoals dimmers, bewegingssensoren, thermostaten en meer, open staan.
