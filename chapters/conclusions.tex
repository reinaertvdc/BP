\chapter{Conclusions \& Future Work} \label{chap:concl}
Having finished our research with a high-fidelity user study in the previous chapter, we are now ready to formulate our conclusions. We review our results and contributions, and we propose possible future work.

Earlier research had already established that AR can play a valuable role in improving the intelligibility of ubicomp environments \cite{bau2008octopocus, vermeulen2009bet, vermeulen2012understanding, vermeulen2013intelligibility}. Our goal was to grow this knowledge base by providing new insights into visualization techniques for spatial awareness. We did this by performing two studies focused specifically on the problem of matching lights and switches, one explorative low-fidelity online survey and one informal high-fidelity user test.

From our explorative survey, we learned that, when no visualizations are present to help match lights and switches, users will try to make a direct mapping between the two. This means they will treat the switches as a map of the room, and assign each light to the switch that ends up closest to it. Barring this, users will assume larger switches control larger or more lights.

The best three visualizations according our survey were, generally speaking, \textit{colors}, \textit{switches on lights} and \textit{lights on switches}. Incidentally, these visualizations are perfectly compatible; we can display the connected lights on each switch, display the connected switch on each light, and do so in a different color for each circuit. This effectively creates a bidirectional dictionary, where each component points to the location of other components, with the distinct colors providing redundancy and helping to find the connected components more easily.

The \textit{shapes} visualization performed equal or worse than \textit{colors} in every way, and we believe it can be considered as just that: an inferior alternative to colors. Participants indicated they take more effort to use and are trusted less. Shapes are also less easily combined with other visualizations than colors. However, shapes may still be useful for the colorblind, they may be combined with colors for increased contrast, and they could even be permutated with colors to increase the number of circuits that can be displayed simultaneously.

The \textit{minimap} visualization has the advantage that it always remains in view, and the \textit{lines} and \textit{arcs} visualizations have the advantage of guiding the user's vision towards the desired target. These attributes could however not be appreciated in a static prototype, which is why we included similar visualizations in our high-fidelity prototype. Here, the minimap became the most popular visualization tool, followed by colors. The combination of the two proved especially effective, as it had users point out different lights in rapid succession without mistakes and without looking up. The two directional indicators, \textit{halo's} and \textit{wedges}, still met with limited success due to our specific implementation, as was argued by nearly every participant. Wedges were by far the more popular of the two, because they create less overlap between indicators, bringing us to the same conclusion that \citeauthor{gustafson2008wedge}, the original creators of Wedge, came to in \citeyear{gustafson2008wedge} \cite{gustafson2008wedge}.

This thesis serves as a first study of the topic, and our implementation remains a proof-of-concept. Possible future work includes creating an improved prototype and running a formal high-fidelity user test, ideally with more participants and in more varied scenarios. We expect that such an experiment would yet prove directional indicators to be effective. We also limited our experimentation to standard lights and light switches, so research into the application of our visualizations with more complex devices, like dimmers, motion sensors, thermostats and so on, remains open.

%- minimap and colors are promising to improve user confidence, trust and understanding, arrows and halo's also work, but might need different scenarios to unlock their true potential, so more testing is needed

%- future work: better graphics would likely improve system effectiveness, small HoloLens FOV is also a handicap, trying out larger and more complex scenario's, other things than lights and switches, more complex interactions than pressing buttons
