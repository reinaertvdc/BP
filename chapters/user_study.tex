\chapter{User Study} \label{chap:user}

\section{Introduction} \label{sec:user:survey:intro}
In the previous chapter, we created a HoloLens application to try out the most promising of our visualizations from the online survey in a more high-fidelity setting. In this chapter, we report how we use the application to conduct an informal user study. We have seven participants try out a range of visualizations, and we collect data by observing the participant's actions and by walking them through a prepared questionnaire. At the and of this chapter, we update our earlier insights based on this newly gathered information.

\section{Procedure} \label{sec:user:survey:procedure}
Each participant was received separately, with only the observer present. The participants were explained that the goal of the experiment was to evaluate various AR visualizations. While wearing the HoloLens, they would be asked six times to match light switches to corresponding lights, each time using a different visualization. They would be asked to think aloud as much as possible, and after each task, they would would answer some questions.

The participants were given some time to adjust the HoloLens straps and get familiar with the device. Then they were asked to look around and confirm they could see virtual overlays on nearby lamps and switches. Next, they were given the clicker and asked to select a switch by gazing at it and clicking, and to confirm that it turned red. After each of these steps were completed, we were confident that the system was working correctly and that the participant understood how to use it.

For each visualization, we told the participant the same words: "Please select a switch, say which one you chose, then show which lights you believe are connected to it. Please try to think aloud as you decide.". We would note the participant's answer and whether it was correct. After each visualization, we asked the participant the same six questions, to which they had to answer on a 1-5 Likert scale:
\begin{enumerate}
  \item How certain are you that the switches work the way you indicated?
  \item How ... do you find the visualization?
  \begin{enumerate}
    \item easy to learn
    \item fast to work with
    \item resistant to errors
    \item useful
    \item enjoyable
  \end{enumerate}
\end{enumerate}

The questions were intended as starting points for conversation, rather than quantitative measurements for statistical analysis, given the limited number of participants. We would ask the participants for the reasoning behind their scores, and if they saw ways to improve the visualizations. Throughout the entire process, we would note any interesting quotes, and also the participant's general behaviour for each visualization, like whether there were long periods of hesitation. After two or three visualizations, we would switch rooms, to introduce some variety and to avoid reusing switches.

Each participant received the same six visualizations, with slight variations in the order:
\begin{enumerate}
  \item World highlights
  \item Minimap highlights
  \item Arrows
  \item Halo's
  \item Combo (minimap highlights combined with arrows or halo's)
  \item Colors
\end{enumerate}

All participants started with the \textbf{world highlights}, which simply change the overlay color of lights from white to red. We considered them the most basic and recognizable visualization, since it has exactly the same effect as actually turning the lights on, but with the downside that the visualization does not radiate light from activated lamps, so users must look straight at them to be able to find them. With a total lack of overview and guidance, we did not consider this visualization an actual contender, at least not by itself. Rather, it was intended as a warm-up, and as a way of giving each participant the same reference point to compare the next visualizations to.

The second visualization for each participant was the \textbf{minimap highlights}, which showed a map of the participant's environment, on which lights would be highlighted in red, in the bottom right corner of their view. In contrast to the previous visualization, the minimap highlights do not require any head movement at all, once the user is acquainted with the environment. We placed this visualization second because maps are again a familiar concept that participants will be used to working with.

The third and fourth visualizations were the directional indicators. Half of the participants used the \textbf{arrows} first, the other half started with the \textbf{halo's}, so as to distribute the learning effect that these similar visualizations may have on each other. We followed up with the \textbf{combo} visualization, which is the minimap highlights combined with whichever of the two directional indicators the participant preferred. With this composite visualization, the participant had multiple ways of finding the correct lights, and we payed careful attention to which way they chose. This let us see how well their stated opinions and their actions lined up. We discussed with the participants whether the visualizations complemented each other, and we asked if they could think of better combinations.

The sixth last visualization for each participant were the \textbf{colors}. This visualization was placed last, because it functioned differently than the others, in the sense that no switch had to be selected. All circuits would be shown at once, each in their own color, both in the world and on the minimap. We pointed to several switches in quick succession and asked the participants to indicate the correct lights for each one. Each time we observed carefully whether the participants looked around, used the minimap, or both, and in what order. We finished with an open-ended discussion where we asked the participant's opinion of the entire concept and asked if they had any more insights to share.

\section{Apparatus} \label{sec:user:survey:apparatus}
Participants wore a HoloLens for the duration of the test, which was about 25 minutes. They used the accompanying clicker to select switches. The environment consisted of two office rooms with the adjacent hallway, and had 8 switches and 22 lights in total.

\section{Participants} \label{sec:user:survey:participants}
There were 7 unpaid male participants, 6 researchers and 1 student. All participants were familiar with the test environment, but none knew the corresponding lights for any specific switch in advance.

\section{Results} \label{sec:user:results}
There were 6 Likert scale questions for 6 visualizations, answered by 7 participants, which makes 42 trials and 252 scores in total, complemented with open questions. We will pay little attention to the scores, since they are inconclusive, and rather focus on the motivations that accompanied them. We can however note that all averages fell between 2.8/5 and 4.9/5, and that most scores were a 4+. We also note that some discussions were not in English. We have translated quotes from these conversations as literally as possible.

Across all 42 trials, 2 wrong answers were given. Both happened with the world highlights, but with different circuits. Both times the participant indicated there were two connected lights when there were actually three, because they missed a light that was hanging almost directly above them. Still, both participants were confident of their answer. In fact, the first question, ``How certain are you that the switches work the way you indicated?'', received a 5/5 in all but one case. Here the halo's received a 4, with the participant calling them ``... less precise because of the overlap. I prefer the arrows, those are more precise, smaller, they don't overlap as much.''.

Before going through the results of each visualizations, we believe it best to present the \textit{useful} and \textit{enjoyable} criteria separately. Most participants indicated they found these questions difficult to answer, especially the latter. Scores ranged widely, even within the same visualization. The negative opinions were mostly based on the HoloLens being unwieldy and having a small projection window, or on the presented problem being simple enough to solve by itself. One participant argued ``It's only lights, in this context nothing is dangerous, so it's not useful, in more complex scenarios it might be useful.'', while another explained that ``I wouldn't call this enjoyable. A football match I would call enjoyable, but turning on the lights I don't do for fun.''. When the scores were low because of the visualization itself, the scores for the other three criteria were also low.

For the \textbf{world highlights}, we observed all participants slowly looking around the room, until they came across the first red highlight, then quickly catching on and finding the other lights. All participants took 5-15 seconds to give their answer, and all described their thought process along the lines of ``I look around, I see red lights, the switch is also red, so these must belong together.''. The participants unanimously agreed the visualization was \textit{easy to learn}, but four participants had second thoughts on them being \textit{fast to work with} or \textit{resistant to errors}, stating that the visualization is ``... just as fast as simply trying out the switches.'' and that ``I can see myself miss a light from time to time.''.

For the \textbf{minimap highlights}, the minimap was not immediately noticed by one participant, who started looking around the room again. Two others saw the map but took some time to understand what it was. The remaining four participants gave the correct answer within seconds, and without looking up to do so, they simply pointed behind their backs. After realizing how the visualization worked, the other three were also quick to answer. Once having responded, four participants tried to confirm their answer by looking up to the lights as they had done previously, but of course there was nothing to see this time around. Six participants found the visualization very \textit{easy to learn}, but one described it as ``... harder than the other one.''. The criteria \textit{fast to work with} and \textit{resistant to errors} again received strong affirmation from all but one participant, who noted ``I think it's weird that part of the room is cut off, that feels like a bug. Some lights are almost cut away, ... but I can see the lights in the room next door, and those are not relevant.'' and ``The map is good for technical people, who are used to working with ground plans, I as an ordinary person don't have much experience with top-down views.".

Between the two directional indicators, six participants preferred the \textbf{arrows}, and one preferred the \textbf{halo's}. The general opinion is well described by this earlier mentioned quote, saying the halo's are ``... less precise because of the overlap. I prefer the arrows, those are more precise, smaller, they don't overlap as much.''. The one participant who preferred the halo's argued they found the concept more familiar. 



- nog steeds mikken en klikken
- note how participants indicate certainty of their answers, but give average scores on "resistant to errors"
- problem with depth, flickering, halo lines should be thinner, replace triangles with actual arrows, indicators disappear and lights turn red when in view

\section{Discussion} \label{sec:user:discussion}
- everyone is confident with every visualization

- nobody made wrong links, but two people missed a light once when using the highlight (and they were confident of their answers)

- everyone loves the minimap, several mention the luxury of not having to move their head. One person said they didn't prefer the minimap, but kept using it, and later changed their opinion.

- Arrows are more popular than halo's, several argue they are less cluttering

- user suggestions: make map larger, replace triangles with arrows

- ...

\section{Conclusions} \label{sec:user:conclusions}
- all visualizations work well, except for simple highlight, which is as we suspected (and hoped)

- minimap + colors + highlight seems generally the best choice, especially in simple scenarios

- arrows and halo's might work better in more complex and larger scenario's, where a minimap becomes to crowded, as is also suggested by several users

- name user suggestions as possible improvements
