\chapter{Exploration} \label{chap:explor}

\section{Introduction} \label{sec:explor:intro}
The previous chapter listed existing interaction techniques combining AR, feedforward and ubicomp. We will now perform our own exploratory study of the design space. The results of this study will help us focus the remainder of our research and prepare for a more high-fidelity experiment in the next chapter. We will base our exploration on an online survey, since it allows for a large number of participants in a short amount of time. See Appendix \ref{chap:explor_survey} for a full printout of the survey, including the introductory information that participants received.

\section{Survey} \label{sec:explor:survey}
    \subsection{Procedure} \label{subsec:explor:survey:procedure}
    We aim to investigate for a range of interaction techniques how well they work in different ubicomp setups. We achieve this by having the participants try out a set of scenarios and rate them on various aspects. Each of these scenarios will have some problem that the participants must try to solve. Some will be baseline scenarios where the participants receive no help, but most will be augmented scenarios where the participants are aided in their task by some form of AR feedforward that we wish to evaluate.

    Since our survey targets the general population, AR and ubicomp may be foreign concepts to some participants, and thus we must take special care not to confuse or overwhelm them. We also prefer the setups to offer problems that the participants have encountered before, helping them to understand the task and stimulating them. For this reason, we opted to focus on the problem of mapping light switches to their corresponding lights. This is a recognizable and conceptually simple problem, but still allows for setups of arbitrary complexity.
    
    Our next question is how many and which setups to choose. Ideally, we would use only one setup for all scenarios, to eliminate an extra variable from our experiment. A second reason to limit the number of setups is that for each setup we must add a baseline scenario without feedforward to compare the other scenarios against. However, the set of possible setups is extremely diverse. A symmetric rectangular hall with a grid of lights on the ceiling might benefit most from different approaches than a cozy lounge with various lights on walls and tables. We could test each approach in multiple setups, but this would reduce the number of approaches we could include.
    
    We decide on using two setups, which limits the number of baselines, but still allows for some variety in the scenarios. We will test each feedforward approach only in the setup in which we expect it to work best, to maximize the number of approaches that we can investigate. This choice reflects the preliminary nature of this study. The promising approaches can still be investigated further in a follow-up study. Both of the chosen setups are of moderate complexity, to create scenarios that challenge our approaches but that one might still encounter in day-to-day life. 
    
    The first setup is the ``Kitchen'', shown in Scenario 1 in Appendix \ref{chap:explor_survey}. It is a picture we found online \cite{FileTROY82:online}. This room has a variety of wall, hanging and ceiling lights, six in total. There are four light switches, one larger than the rest, arranged in two rows. This setup is meant to be the more complex of the two. We chose it because we believe there is no single obvious mapping of switches on lights. We might create matches based on similar locations, sizes, or some other metric, but in many cases these metrics will contradict each other. This conflict is intentional, as it allows us to determine which metrics the participants will choose over others.
    
    The second setup is the ``Classroom'', a highly symmetric setup comprised of a rectangular room with a 3x3 grid of lights on the ceiling and a row of 3 switches, one large and two small, next to the door. This room does actually exist, and 
    
    The survey used a within-subject design. The independent variable was the type of feedforward
    \todo{}

    \subsection{Apparatus} \label{subsec:explor:survey:apparatus}
    The survey was created and hosted using Google Forms. It was made available via several online media for the duration of one week, from 2017-11-08 until 2017-11-15. We estimate the average completion time at 20 minutes.

    \subsection{Participants} \label{subsec:explor:survey:participants}
    The survey received a total of 36 unpaid responses. After careful consideration, we chose to discard two responses. In the first discarded submission, the participant wrote that they could not answer some questions due to technical difficulties, and in the second, the participant's commentary clearly showed that they had misunderstood some questions. Our results and analysis are thus based on 34 responses, with demographics as shown in Table \ref{table:explor:demographics}. The average age of the participants was 27 years.

        \begin{table}[h!]
        \centering
            \begin{tabular}{|c|c c|c|} 
            \hline
                      & Male & Female &    \\
            \hline
            Age 17-25 &   18 &      8 & 26 \\
            Age 33-53 &    5 &      3 &  8 \\
            \hline
                      &   23 &     11 & 34 \\
            \hline
            \end{tabular}
        \caption{Demographics}
        \label{table:explor:demographics}
        \end{table}

    

\section{Results} \label{sec:explor:results}
\todo{}

\section{Discussion} \label{sec:explor:discussion}
\todo{}