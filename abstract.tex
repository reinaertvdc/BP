\begin{abstract}
    Ubiquitous computing and augmented reality are quickly coming of age. The falling prices of smartphones and other powerful hardware creates a wide platform, that is used by software like digital assistants and Pokémon Go to introduce the big public to these new concepts in an affordable way. Devices like the Microsoft HoloLens, which allow for fast prototyping of both technologies, are now available to developers.
    
    Each emerging technology brings their own new challenges, and an important challenge in ubiquitous computing concerns the fact that actions in the real world are often irreversible, in contrast with traditional computing. Therefore, feedforward, the process of telling the user what the result of an action will be before they execute it, becomes essential rather than just valuable. We theorized that AR can be an excellent tool for providing this feedforward.
    
    While we considered many different uses of feedforward through AR in this thesis, we chose to focus on the problem of telling the user which lights will turn on when they press a particular light switch. This problem is recognizable and easy to understand, which facilitates experimentation, while still providing sufficient complexity.
    
    After the initial literature study, we started our research with an exploration of the design space, based on an online survey with 34 responses. In this survey, we evaluated seven different AR visualizations that help with matching lights and switches. We concluded that maps and colors are the most promising approaches. While the two directional indicator visualizations, being lines and arcs, did not score particularly well, we attributed this to the static and low-fidelity nature of the survey, and suspected that they would perform much better in a more high-fidelity experiment.
    
    Our final step was to implement feedforward via maps, colors and directional indicators on a Microsoft HoloLens, and to evaluate our implementation in an informal user study with seven participants. The results of this study strengthened our earlier suspicions that maps and colors are excellent ways to solve the matching problem of lights and switches, but the directional indicators scored mixed results, with some negativity attributed to our specific implementation and some to the provided scenarios.
    
    Possible future work includes creating an improved prototype and running a formal high-fidelity user test, ideally with more participants and in more varied scenarios. We suspect that such an experiment would yet prove directional indicators to be effective. We also limited our experimentation to standard lights and light switches, so research into the application of our visualizations with more complex devices, like dimmers, motion sensors, thermostats and so on, remains open.
\end{abstract}
